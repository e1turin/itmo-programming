%%%%%%%%%%%%%%%%%%%%%%%%%%%%%%%%% LAB-5 %%%%%%%%%%%%%%%%%%%%%%%%%%%%%%%%%%
%>>>>>>>>>>>>>>>>>>>>>>>>>> ПЕРЕМЕННЫЕ >>>>>>>>>>>>>>>>>>>>>>>>>>>>>>>>>>>
%>>>>> Информация о кафедре
%\newcommand{\year}{2021 г.}  % Год устанавливается автоматически
\newcommand{\city}{Санкт-Петербург}  %  Футер, нижний колонтитул на титульном листе
\newcommand{\university}{Национальный исследовательский университет ИТМО}  % первая строка
\newcommand{\department}{Факультет программной инженерии и компьютерной техники}  % Вторая строка
\newcommand{\major}{Направление системного и прикладного программного обеспечения}  % Треьтя строка
%<<<<< Информация о кафедре

%>>>>> Назание работы
\newcommand{\lab}{Лабораторная работа}
\newcommand{\labnumber}{№ 4}                    % порядковый номер работы
\newcommand{\subject}{Программирование}         % учебный предмет
\newcommand{\labtheme}{Принципы ООП}            % Тема лабораторной работы
\newcommand{\variant}{№ 1234510.5}                % номер варианта работы

\newcommand{\student}{Тюрин Иван Николаевич}    % определение ФИО студента
\newcommand{\studygroup}{P3110}                 % определение учебной группы 
\newcommand{\teacher}{Письмак А. Е.,\\[1mm]     % ФИО лектора
                        Сорокин Р. Б.}          % ФИО практика
%<<<<<<<<<<<<<<<<<<<<<<<<<< ПЕРЕМЕННЫЕ <<<<<<<<<<<<<<<<<<<<<<<<<<<<<<<<<<<


%>>>>>>>>>>>>>>>>>>>>>> ПРЕАМБУЛА >>>>>>>>>>>>>>>>>>>>>>>>>
\include{preamble}
%<<<<<<<<<<<<<<<<<<<<<< ПРЕАМБУЛА <<<<<<<<<<<<<<<<<<<<<<<<<


%%%%%%%%%%%%%%%%%%% СОДЕРЖИМОЕ ОТЧЕТА %%%%%%%%%%%%%%%%%%%%%
%>>>>>>>>>>>>>>> ''''''''''''''''''''''' >>>>>>>>>>>>>>>>>>
\begin{document}


%>>>>>>>>>>>>>>>> ОПРЕДЕЛЕНИЕ НАЗВАНИЙ >>>>>>>>>>>>>>>>>>>>
% Переоформление некоторых стандартных названий
%\renewcommand{\chaptername}{Лабораторная работа}
\renewcommand{\chaptername}{\lab\ \labnumber} % переименование глав
\def\contentsname{Содержание} % переименование оглавления
%<<<<<<<<<<<<<<<< ОПРЕДЕЛЕНИЕ НАЗВАНИЙ <<<<<<<<<<<<<<<<<<<<


%>>>>>>>>>>>>>>>>> ТИТУЛЬНАЯ СТРАНИЦА >>>>>>>>>>>>>>>>>>>>>
\include{titlepage}
%<<<<<<<<<<<<<<<<< ТИТУЛЬНАЯ СТРАНИЦА <<<<<<<<<<<<<<<<<<<<<


%>>>>>>>>>>>>>>>>>>>>> СОДЕРЖАНИЕ >>>>>>>>>>>>>>>>>>>>>>>>>
% Содержание
\tableofcontents
%<<<<<<<<<<<<<<<<<<<<< СОДЕРЖАНИЕ <<<<<<<<<<<<<<<<<<<<<<<<<


%%%%%%%%%%%%%%%%%%%%%%% КОД РАБОТЫ %%%%%%%%%%%%%%%%%%%%%%%%
%>>>>>>>>>>>>>>>>>>>'''''''''''''''''>>>>>>>>>>>>>>>>>>>>>
\newpage
\Chapter{\lab\ \labnumber}{\labtheme}

\Section{Задание варианта \variant}
\begin{center}
, , ,
\end{center}
\noindent
\textbf{
    Описание предметной области, по которой должна быть построена объектная модель:
}

\textit{
    В это время командир полицейского отряда Ригль приложил ко рту руки рупором и закричал издали: Заметив, что полицейские взяли на изготовку ружья, Знайка скомандовал коротышкам: Пропустив вперед Фуксию и Селедочку, коротышки один за другим полезли в ракету. Послышались выстрелы. Вокруг засвистали пули. Клепка, обычно оказывавшийся впереди всех, но на этот раз оказавшийся позади, почувствовал вдруг, как что-то обожгло ему руку чуть повыше локтя. Знайка, который решил сесть в ракету последним, увидел, как лицо Клепки исказилось от боли, а на белом рукаве рубашки появилось красное расплывающееся пятно крови. Схватив Клепку в охапку, Знайка втащил его в кабину и, не теряя ни секунды, захлопнул за собой дверь. Доктор Пилюлькин увидел, что Клепка ранен, и бросился к нему со своей походной аптечкой. Осмотрев рану и установив, что пуля прошла навылет, не задев кость, Пилюлькин быстро остановил кровотечение и наложил на рану повязку. Клепка терпеливо переносил боль. Услышав, что пули так и барабанят по стальной оболочке ракеты, Знайка посмотрел в иллюминатор. Полицейские продолжали беспорядочную стрельбу.\\
}

\noindent
\textbf{
    Программа должна удовлетворять следующим требованиям:
}
\begin{enumerate}
    \item В программе должны быть реализованы 2 собственных класса исключений (\texttt{checked} и \texttt{unchecked}), а также обработка исключений этих классов.
    \item В программу необходимо добавить использование локальных, анонимных и вложенных классов (\texttt{static} и \texttt{non-static}).
\end{enumerate}
\begin{center}
    ' ' '
\end{center}

\newpage
\Section{Выполнение задания}
В результате выполнения работы по приведенному техническому заданию была доработана объектная моодель приложения, перерисована UML диаграмма классов (см. \ref{fig:uml}) в соответствии с внесенными изменениями. Модифицирована программа чтобы соответствовать объектной модели. 

\Section{UML диаграмма классов}
%      |   <Имя файла>   |Отн. ширина|    Описание картики  |Тег (fig:uml)|
\Figure{UML-class-diagram}{\textwidth}{UML диаграмма классов}{uml}


\Section{Исходный код программы}
Исходный код программы был размещен на удаленном сервере в личном репозитории. Код можно найти по ссылке: \url{https://github.com/e1turin/itmo-programming/tree/main/lab-4/src}.


\Section{Результат работы программы}
\begin{center}
, , ,
\end{center}
\begin{verbatim}
Не верное имя файла, файл не обнаружен
не передано имя
введите имя в качестве аргумента командной строки
*Имя коротышек установлено по-умолчанию*
Ригль скомандавал(а) Полицейские, ружья на изготовку
Знайка увидел(а) как Ригль скомандавал(а) Полицейские, ружья на изготовку
Знайка скомандовал(а) Коротышки: забираться в ракету
Знайка решил(а) сесть в ракету
Коротышки пропустил(а) Фуксия
Коротышки пропустил(а) Селедочка
Выстрелы: Звучит Выстрелы
Пули: Звучит Свист

Клёпка почувствовал(а) как Что-то обожгло Клёпка а именно: руку
Клёпка исказил(а) лицо от чувства Боль
Знайка увидел(а) как Клёпка исказил лицо от чувства Боль
Знайка увидел(а) как Кровяное пятно цвета Красный появилось на Рукав цвета Белый
Знайка втащил(а) Клёпка
не теряя ни секунды Дверь Неопределеныйцвета закрылась. Звучит Стук
java.lang.ArrayIndexOutOfBoundsException: Index 0 out of bounds for length 0
	at com.company.Inputter.inputName(Main.java:104)
	at com.company.Main.main(Main.java:28)
Закрыли дверь
Знайка закрыл(a)
Доктор Пилюлькин увидел(а) как Что-то обожгло Клёпка а именно: руку
Доктор Пилюлькин бросился(лась) к Клёпка с Походная аптечка
Доктор Пилюлькин увидел(а) как Пуля прошла навылет
Доктор Пилюлькин подлечил Клёпка
Клёпка почувствовал(а) Боль
Знайка услышал(а), как Пули барабанят по оболочке Рокета
\end{verbatim}
\begin{center}
' ' '
\end{center}

\Section{Вывод}
Изучил принцип программирования SOLID. Укрепил знания по работе с ЯП Java, получил больше знаний об парадигме ООП представленной в ЯП Java. Получил опыт выполнения технического задания. Изучил принципы работы с исключениями в ЯП Java. Создал и обработал собственные checked и unchecked исключения.\\
\newpage
%<<<<<<<<<<<<<<<<<<<<<< КОД РАБОТЫ <<<<<<<<<<<<<<<<<<<<<<<<


%>>>>>>>>>>>>>>>> СПИСОК ЛИТЕРАТУРЫ >>>>>>>>>>>>>>>>>>>>>>>
%\include{biblist}  % Для соответсвия гост, придется доработать. Нужен файл .bib
%<<<<<<<<<<<<<<<<<<<< СПИСОК ЛИТЕРАТУРЫ <<<<<<<<<<<<<<<<<<<


\end{document}
%<<<<<<<<<<<<<<<< ,,,,,,,,,,,,,,,,,,,,,,, <<<<<<<<<<<<<<<<<
%<<<<<<<<<<<<<<<<<<< СОДЕРЖИМОЕ ОТЧЕТА <<<<<<<<<<<<<<<<<<<<
