%%%%%%%%%%%%%%%%%%%%%%%%%%%%%%%%% LAB-5 %%%%%%%%%%%%%%%%%%%%%%%%%%%%%%%%%%
%>>>>>>>>>>>>>>>>>>>>>>>>>> ПЕРЕМЕННЫЕ >>>>>>>>>>>>>>>>>>>>>>>>>>>>>>>>>>>
%>>>>> Информация о кафедре
%\newcommand{\year}{2021 г.}  % Год устанавливается автоматически
\newcommand{\city}{Санкт-Петербург}  %  Футер, нижний колонтитул на титульном листе
\newcommand{\university}{Национальный исследовательский университет ИТМО}  % первая строка
\newcommand{\department}{Факультет программной инженерии и компьютерной техники}  % Вторая строка
\newcommand{\major}{Направление системного и прикладного программного обеспечения}  % Треьтя строка
%<<<<< Информация о кафедре

%>>>>> Назание работы
\newcommand{\lab}{Лабораторная работа}
\newcommand{\labnumber}{№ 5}                    % порядковый номер работы
\newcommand{\subject}{Программирование}         % учебный предмет
\newcommand{\labtheme}{Принципы ООП}            % Тема лабораторной работы
\newcommand{\variant}{№ 236587}                % номер варианта работы

\newcommand{\student}{Тюрин Иван Николаевич}    % определение ФИО студента
\newcommand{\studygroup}{P3110}                 % определение учебной группы 
\newcommand{\teacher}{Письмак А. Е.,\\[1mm]     % ФИО лектора
                        Сорокин Р. Б.}          % ФИО практика
%<<<<<<<<<<<<<<<<<<<<<<<<<< ПЕРЕМЕННЫЕ <<<<<<<<<<<<<<<<<<<<<<<<<<<<<<<<<<<


%>>>>>>>>>>>>>>>>>>>>>> ПРЕАМБУЛА >>>>>>>>>>>>>>>>>>>>>>>>>
\include{preamble}
%<<<<<<<<<<<<<<<<<<<<<< ПРЕАМБУЛА <<<<<<<<<<<<<<<<<<<<<<<<<


%%%%%%%%%%%%%%%%%%% СОДЕРЖИМОЕ ОТЧЕТА %%%%%%%%%%%%%%%%%%%%%
%>>>>>>>>>>>>>>> ''''''''''''''''''''''' >>>>>>>>>>>>>>>>>>
\begin{document}


%>>>>>>>>>>>>>>>> ОПРЕДЕЛЕНИЕ НАЗВАНИЙ >>>>>>>>>>>>>>>>>>>>
% Переоформление некоторых стандартных названий
%\renewcommand{\chaptername}{Лабораторная работа}
\renewcommand{\chaptername}{\lab\ \labnumber} % переименование глав
\def\contentsname{Содержание} % переименование оглавления
%<<<<<<<<<<<<<<<< ОПРЕДЕЛЕНИЕ НАЗВАНИЙ <<<<<<<<<<<<<<<<<<<<


%>>>>>>>>>>>>>>>>> ТИТУЛЬНАЯ СТРАНИЦА >>>>>>>>>>>>>>>>>>>>>
\include{titlepage}
%<<<<<<<<<<<<<<<<< ТИТУЛЬНАЯ СТРАНИЦА <<<<<<<<<<<<<<<<<<<<<


%>>>>>>>>>>>>>>>>>>>>> СОДЕРЖАНИЕ >>>>>>>>>>>>>>>>>>>>>>>>>
% Содержание
\tableofcontents
%<<<<<<<<<<<<<<<<<<<<< СОДЕРЖАНИЕ <<<<<<<<<<<<<<<<<<<<<<<<<


%%%%%%%%%%%%%%%%%%%%%%% КОД РАБОТЫ %%%%%%%%%%%%%%%%%%%%%%%%
%>>>>>>>>>>>>>>>>>>>'''''''''''''''''>>>>>>>>>>>>>>>>>>>>>
\newpage
\Chapter{\lab\ \labnumber}{\labtheme}

\Section{Задание варианта \variant}
\begin{center}
, , ,
\end{center}
\noindent
\textbf{
    Описание предметной области, по которой должна быть построена объектная модель:
}

Реализовать консольное приложение, которое реализует управление коллекцией объектов в интерактивном режиме. В коллекции необходимо хранить объекты класса \verb|MusicBand|, описание которого приведено ниже.\\

Разработанная программа должна удовлетворять следующим требованиям:\\
\begin{itemize}
\setlength{\itemsep}{0pt}
    \setlength{\parskip}{0pt}
    \setlength{\parsep}{0pt}
\item Класс, коллекцией экземпляров которого управляет программа, должен реализовывать сортировку по умолчанию.

\item Все требования к полям класса (указанные в виде комментариев) должны быть выполнены.
\item Для хранения необходимо использовать коллекцию типа \verb|java.util.LinkedHashSet|
\item При запуске приложения коллекция должна автоматически заполняться значениями из файла.
\item Имя файла должно передаваться программе с помощью: аргумент командной строки.
\item Данные должны храниться в файле в формате \verb|json|
\item Чтение данных из файла необходимо реализовать с помощью класса \verb|java.io.InputStreamReader|
\item Запись данных в файл необходимо реализовать с помощью класса \verb|java.io.PrintWriter|
\item Все классы в программе должны быть задокументированы в формате \verb|javadoc|.
\item Программа должна корректно работать с неправильными данными (ошибки пользовательского ввода, отсутсвие прав доступа к файлу и т.п.).
\end{itemize}

\noindent
В интерактивном режиме программа должна поддерживать выполнение следующих команд:
\begin{itemize}
\setlength{\itemsep}{0pt}
    \setlength{\parskip}{0pt}
    \setlength{\parsep}{0pt}
\item \verb|help| : вывести справку по доступным командам
\item \verb|info| : вывести в стандартный поток вывода информацию о коллекции (тип, дата инициализации, количество элементов и т.д.)
\item \verb|show| : вывести в стандартный поток вывода все элементы коллекции в строковом представлении
\item \verb|add {element}| : добавить новый элемент в коллекцию
\item \verb|update id {element}| : обновить значение элемента коллекции, id которого равен заданному
\item \verb|remove_by_id id| : удалить элемент из коллекции по его id
\item \verb|clear| : очистить коллекцию
\item \verb|save| : сохранить коллекцию в файл
\item \verb|execute_script file_name| : считать и исполнить скрипт из указанного файла. В скрипте содержатся команды в таком же виде, в котором их вводит пользователь в интерактивном режиме.
\item \verb|exit| : завершить программу (без сохранения в файл)
\item \verb|add_if_max {element}| : добавить новый элемент в коллекцию, если его значение превышает значение наибольшего элемента этой коллекции
\item \verb|remove_greater {element}| : удалить из коллекции все элементы, превышающие заданный
\item \verb|history| : вывести последние 6 команд (без их аргументов)
\item \verb|average_of_number_of_participants| : вывести среднее значение поля numberOfParticipants для всех элементов коллекции
\item \verb|count_less_than_albums_count albumsCount| : вывести количество элементов, значение поля \verb|albumsCount| которых меньше заданного
\item \verb|print_ascening| : вывести элементы коллекции в порядке возрастания
\end{itemize}

\noindent
Формат ввода команд:
\begin{itemize}
\setlength{\itemsep}{0pt}
    \setlength{\parskip}{0pt}
    \setlength{\parsep}{0pt}
\item Все аргументы команды, являющиеся стандартными типами данных (примитивные типы, классы-оболочки, \verb|String|, классы для хранения дат), должны вводиться в той же строке, что и имя команды.
\item Все составные типы данных (объекты классов, хранящиеся в коллекции) должны вводиться по одному полю в строку.
\item При вводе составных типов данных пользователю должно показываться приглашение к вводу, содержащее имя поля (например, "Введите дату рождения:")
\item Если поле является enum'ом, то вводится имя одной из его констант (при этом список констант должен быть предварительно выведен).
\item При некорректном пользовательском вводе (введена строка, не являющаяся именем константы в \verb|enum|'е; введена строка вместо числа; введённое число не входит в указанные границы и т.п.) должно быть показано сообщение об ошибке и предложено повторить ввод поля.
\item Для ввода значений \verb|null| использовать пустую строку.
\item Поля с комментарием "Значение этого поля должно генерироваться автоматически" не должны вводиться пользователем вручную при добавлении.
\end{itemize}

\noindent
Описание хранимых в коллекции классов:
\refstepcounter{lstlisting}
\begin{figure}[h] %- \usepackage {float} %[h]
    \begin{center}
        \lstinputlisting[language=Java]{task-class-description.java}
    \end{center}
    \captionof{lstlisting}{Описание хранимых в коллекции классов}
    \label{lst:objects}
\end{figure}

\begin{center}
    ' ' '
\end{center}

\newpage
\Section{Выполнение задания}
В результате выполнения работы по приведенному техническому заданию была разработана объектная моодель приложения, нарисована UML диаграмма классов (см. \ref{fig:uml}).

% \Section{UML диаграмма классов}
%      |   <Имя файла>   |Отн. ширина|    Описание картики  |Тег (fig:uml)|
\Figure{Classes}{\textwidth}{UML диаграмма классов}{uml}


\Section{Исходный код программы}
Исходный код программы был размещен на удаленном сервере в личном репозитории. Код можно найти по ссылке: \url{https://github.com/e1turin/itmo-programming/lab-5-kotlin}.


\Section{Вывод}
Укрепил знание принципов программирования SOLID и инъециорования зависимостей. Укрепил знания по работе с ЯП Java, изучил ЯП Kotlin, получил больше знаний о парадигме ООП и ФП представленной в ЯП Kotlin. Получил еще один опыт выполнения технического задания. Научился основам конфигурирования сборщика проектов Gradle. Научился сдерживать ярость, несколько раз упсиховался, научился быстро писать отчет к работе, узнал на каких сайтах хороший код и откуда можно безопасно копипастить. Понял отличия между библиотеками по работе с JSON: Jackson и Gson (второй круче). Не понял зачем перед считыванием целой строки в жаве нужно предварительно считать пустую строку. Возможно научился основам упаковки проета в контейнеры Docker (если успею сделать до сдачи).\\
\newpage
%<<<<<<<<<<<<<<<<<<<<<< КОД РАБОТЫ <<<<<<<<<<<<<<<<<<<<<<<<


%>>>>>>>>>>>>>>>> СПИСОК ЛИТЕРАТУРЫ >>>>>>>>>>>>>>>>>>>>>>>
%\include{biblist}  % Для соответсвия гост, придется доработать. Нужен файл .bib
%<<<<<<<<<<<<<<<<<<<< СПИСОК ЛИТЕРАТУРЫ <<<<<<<<<<<<<<<<<<<


\end{document}
%<<<<<<<<<<<<<<<< ,,,,,,,,,,,,,,,,,,,,,,, <<<<<<<<<<<<<<<<<
%<<<<<<<<<<<<<<<<<<< СОДЕРЖИМОЕ ОТЧЕТА <<<<<<<<<<<<<<<<<<<<
